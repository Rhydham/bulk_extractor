
%\documentclass[10pt,twoside,twocolumn]{article}
\documentclass[10pt,twoside]{article}
\usepackage[bf,small,nooneline]{caption}
\usepackage[letterpaper,hmargin=1in,vmargin=1in]{geometry}
\usepackage{paralist} % comapctitem, compactdesc, compactenum
\usepackage{titlesec}
\usepackage{titletoc}
\usepackage{times}
\usepackage{hyperref}
%\usepackage{glossaries}
%\usepackage[xindy]{glossaries}
\usepackage[toc]{glossaries}
\usepackage{graphicx}
\graphicspath{{./graphics/}}
\usepackage{xspace}
\usepackage{verbatim}
\hyphenation{Sub-Bytes Shift-Rows Mix-Col-umns Add-Round-Key}

\newcommand{\bulk}{\emph{bulk\_extractor}\xspace}
\newcommand{\bev}{\emph{BEViewer}\xspace}
\newcommand{\button}[1]{(\raisebox{-1.20mm}{\includegraphics[scale=0.90]{./graphics/#1}})}

\newglossaryentry{BEViewer}
{name={\emph{BEViewer}},description={The Bulk Extractor Viewer User Interface application}}

\newglossaryentry{BulkExtractor}
{name={\emph{bulk\_extractor}},description={The utility for extracting Features from Image media.
\bulk also reads embedded Paths for \bev}}

\newglossaryentry{Feature}
{name={Feature}, description={A sequence of bytes that are specifically searched for
and found in an Image during a \bulk scan}}

\newglossaryentry{FeatureFile}
{name={Feature File}, description={A File containing Features, created during a \bulk scan}}

\newglossaryentry{HistogramFile}
{name={Histogram File}, description={A File containing a histogram of Features
produced from its associated Feature File, created during a \bulk scan}}

\newglossaryentry{Histogram}
{name={Histogram}, description={The number of occurrences (frequency)
of various Features in a Feature File}}

\newglossaryentry{ReferencedFeatureFile}
{name={Referenced Feature File}, description={A Feature File that is referenced to
by its associated Histogram file.
For example the referenced Feature File for file \texttt{email\_histogram.txt}
is \texttt{email.txt}.
\bev displays the Referenced Feature File when a Histogram file is selected in the Reports View}}

\newglossaryentry{ReferencedFeature}
{name={Referenced Feature}, description={The Features contained within a Referenced Feature File}}

\newglossaryentry{Bookmark}
{name={Bookmark}, description={To add a Feature to the set of Bookmarked Features
so that it can be saved in a Case or placed in a Report}}

\newglossaryentry{BookmarkedFeature}
{name={Bookmarked Feature}, description={A Feature that has been Bookmarked.
Bookmarked Features may be exported as Case Evidence}}

\newglossaryentry{Report}
{name={Report}, description={The Feature Files and associated information
generated by running \bulk.
\bev opens Report directories created by \bulk
so that Features in their Feature Files may be Navigated to}}

\newglossaryentry{Case}
{name={Case}, description={The collection of Reports and Bookmarks
associated with a Forensics Investigation performed by an Examiner using \bev.
Cases may be opened and saved}}

\newglossaryentry{Navigate}
{name={Navigate}, description={To show a Feature, its attributes, and a page of its Image View
in the Navigation Area}}

\newglossaryentry{NavigationHistory}
{name={Navigation History}, description={The collection of Features that have been Navigated to
during the \bev session.
The list of Features is lost when \bev closes.
To make Features persistent, Bookmark them}}

\newglossaryentry{Path}
{name={Path}, description={The location of a Feature in an Image.
Direct paths are represented by an offset into the Image.
Compressed paths are represented by syntax indicating how the Feature resides in compressed form,
and includes the offset into the decompressed space}}

\newglossaryentry{Offset}
{name={Offset}, description={The location into an Image
where a Feature or compressed region resides}}

\newglossaryentry{ImageFile}
{name={Image File}, description={A file containing a Forensics Image from digital media.
Image Files may be contained in any of several Forensics formats such as \texttt{.E01},
\texttt{.001}, and {.aff}}}

\newglossaryentry{Image}
{name={Image}, description={The actual Image media, regardless of the file format that contains it}}

\newglossaryentry{AddressBase}
{name={Address Base}, description={The numeric base used for displaying Offset values,
specificallly Decimal or Hexadecimal}}

\newglossaryentry{Highlight}
{name={Highlight}, description={To color the background around Feature text and Image text
to make it more visible.
Feature text or typed text may be highlighted,
depending on the Highlight Source selected by the user}}

\newglossaryentry{HighlightSource}
{name={Highlight Source}, description={The source to use for highlighting text,
either the Navigated Feature or else typed text}}

\newglossaryentry{Log}
{name={Log}, description={A record of user actions and internal operations maintained by \bev
useful for diagnostics in the event of an error}}

\newglossaryentry{Filter}
{name={Filter}, description={Text used for controlling the list of Features shown
in the Features View.
Only Features matching this text are displayed.
Sensitivity to capitalization may also be set}}

\newglossaryentry{SystemClipboard}
{name={System Clipboard}, description={A System scratch pad used for copying
Features in the Features Area and Image bytes in the Navigation Area
from \bev into other applications running on the system}}

\newglossaryentry{ImageReader}
{name={Image Reader}, description={The internal reader that \bev uses for reading Image bytes
from Image Files.
Although \bev selects the appropriate Image Reader to use,
a specific Image Reader may be set for testing purposes}}

\makeglossaries

\begin{document}

%\titlespacing{\section}{0pt}{0pt}{0pt}
%\titlespacing{\subsection}{0pt}{0pt}{0pt}
%\titlespacing{\subsubsection}{0pt}{0pt}{0pt}
%\date{}

\title{Bulk Extractor Viewer (\bev) User Manual}
\author{Bruce Allen \footnote{\href{mailto:bdallen@nps.edu}{bdallen@nps.edu}}}
\maketitle

%\begin{abstract}
%  The Bulk Extractor Viewer (\bev)
%  is a User Interface for browsing Feature data extracted
%  using the \bulk extraction tool.
%  In addition to browsing Features,
%  \bev provides facilities for viewing multiple Images,
%  managing Feature sets as Cases,
%  and exporting Bookmarked Features.
%\end{abstract}

%\newpage
\cleardoublepage
\setcounter{tocdepth}{2}
\tableofcontents
%\newpage
\cleardoublepage

\section{Introduction}
\bulk and \bev together provide rapid Forensics triage of digital media.
\bulk rapidly identifies \glspl{Feature} in a Forensics \gls{Image}
to produce \glspl{FeatureFile} containing \glspl{Feature} entries
obtained from various scanners that \bulk runs.
\bulk also produces \glspl{HistogramFile} of \gls{Histogram} entries
indicating the frequency of specific Features.

\bev provides a User Interface for browsing through Features identified by \bulk
to quickly find the presence of specific Feature content.
To facilitate this effort, \bev provides feature filtering, highlighting,
and the ability to print, export \glspl{Case} Features, and manage multiple \glspl{Report}.

\subsection{\bulk Capabilities}
\bulk scans Forensic \glspl{Image} to produce \glspl{Feature}.
\bulk also provides facilities for reading embedded media
such as content within an embedded \texttt{.zip} file.

\bulk provides the following capabilities:
\begin{compactitem}

\item \bulk Captures \glspl{Feature} of various types:
% this compactitem list is copied from bulk_extractor/bulk_extractor.tex
\begin{compactitem}
\item Email addresses
\item Credit card numbers, including track 2 information
\item Search terms (extracted from URLs)
\item Phone numbers
\item GPS coordinates
\item EXIF (Exchangeable Image File Format) information from JPEG
  images
\item A list of all words present on the disk, for use in
  password cracking
\end{compactitem}

\item \bulk runs on Windows, Linux and Macintosh-based systems.
\item \bulk operates on raw disk images, split-raw volumes, EnCase E01 files, and AFF files.
\item \bulk additionally extracts features
from compressed data such as ZIP and windows hibernation files.
\item \bulk employs Thread Pool optimization
which fully utilizes each processor on the system to maximize processing performance.
\item \bulk can extract regular expressions provided as user input.
\item \bulk runs plug-ins for custom Image processing.
\item \bulk recognizes context-sensitive stop-lists.
\end{compactitem}

\subsection{\bev Capabilities}
The \gls{BEViewer} User Interface allows the user
to browse through \glspl{Feature} in \glspl{FeatureFile} created
using the \gls{BulkExtractor} tool \cite{garfinkel:bulk-extractor}.
\bev also provides interfaces for managing \glspl{Case}, generating \glspl{Report},
and running \bulk scans.

\bev provides the following capabilities:
\begin{compactitem}
\item Browse \glspl{Feature}:
\begin{compactitem}
\item View Features in \glspl{FeatureFile} created by \bulk.
\item Scroll the \gls{Image} view to see Features in their context.
\item View Features from multiple Images.
%\item Open multiple \glspl{Report} from multiple \glspl{Image} simultaneously.
\item \gls{Navigate} between Features using a History list.
\item \gls{Bookmark} features for including them in Cases and Reports.
\item \gls{Highlight} Feature content for easy visual identification.
\item Copy Feature and Image content to the \gls{SystemClipboard}.
\item Print Image content directly.
\item Filter Features to find specific text.
\end{compactitem}
\item Manage \glspl{Case}:
\begin{compactitem}
\item Bookmark Features to include them in a Case.
\item Save and open Case settings as Case files.
\item Transfer Case files to other Examiners or to another computer.
\end{compactitem}
\item Generate \glspl{Report}:
\begin{compactitem}
\item Bookmark Features to be placed in Reports.
\item Manage the Bookmarked Features list.
\item Export Bookmarked Features to a file for reporting.
\end{compactitem}
\item Run \gls{BulkExtractor}:
\begin{compactitem}
\item Launch the \bulk extraction tool.
\item Monitor the progress of \bulk as it runs.
\end{compactitem}
\end{compactitem}

\subsection{Obtaining \bulk and \bev}
\bev requires \bulk to produce extracted \gls{Feature} data.
Please obtain \bev and \bulk from \url{https://github.com/simsong/bulk\_extractor}.
For instructions on installing \bulk please see https://github.com/simsong/bulk\_extractor/wiki/.

\subsection{Running \bulk and \bev}
Please run \texttt{BEViewer.jar} from a command prompt
by typing \texttt{java -Xmx1g -jar BEViewer.jar}.
The \texttt{-Xmx1g} parameter runs \bev with more memory,
which is necessary when acessing large Feature files.

Although \bulk may be run directly, it may readily be run from the \bev UI.
If you would like to run \bulk directly from a command line,
please see https://github.com/simsong/bulk\_extractor/wiki/.

\section{\bev Interfaces}
An example main window of an active \bev session is shown in Figure~\ref{main-window-example}.
\begin{figure}
\center
\includegraphics[scale=0.5]{graphics/ExampleView}
\caption{An example \bev view with Reports opened, Feature Files loaded,
and a Feature navigated to.\label{main-window-example}}.
\end{figure}
An empty \bev window, with no Reports loaded, and with areas annotated,
is shown in Figure~\ref{main-window-parts}.
\begin{figure}
\center
\includegraphics[scale=0.6]{graphics/PartsView}
\caption{The \bev view with its areas labeled.\label{main-window-parts}}
\end{figure}

This section defines the areas of the \bev main window.
\subsection{Menu Bar}
The Menu Bar contains the menus that control \bev.
\begin{compactitem}
\item Use the \emph{File} menu to manages Reports, Bookmarks, and Cases.
\item Use the \emph{Edit} menu to copy selections to the System Clipboard
and to clear the Navigation history.
\item Use the \emph{View} menu to manage View settings and observe miscellaneous properties.
\item Use the \emph{Tools} menu to run \bulk.
\item Use the \emph{Help} menu to obtain Version information and Help information,
view \glspl{Log},
and manage diagnostics.
\end{compactitem}

\subsection{Reports Area}
The Reports Area contains the Reports View which contains a list of the loaded Reports.

\subsubsection{Reports View}
The Reports View contains the list of loaded Reports shown in outline view.
When the outline view is expanded, the view shows the list of Feature Files
that \bulk created when it was run.

Control the Reports View as follows:
\begin{compactitem}
\item Click the outline triangle on a Report in the Reports View to expand or contract
the list of Feature Files associated with the given Report.
\item Click on a Feature File to select it and see its Features listed in the Features Area.
\end{compactitem}

\subsection{Features Area}
The Features Area contains a list of the Features associated with the Feature File
that is selected in the Reports View.
It also contains the \gls{Filter} control for filtering which Feature entries are viewed.

\subsubsection{Filter Control}
The Filter Control area allows you to filter the Feature entries
so that they show only the Features you request.

Control filtering as follows:
\begin{compactitem}
\item Click the capitalization case button to ignore \button{CaseInsensitive}
or honor \button{CaseSensitive} capitalization case sensitivity for filtered text.
\item Type text in the text entry field to provide the Feature filter text.
\end{compactitem}

\subsubsection{Features View}
The Features View shows the Feature entries from the Feature File
that is selected in the Reports View.
The Feature list is filtered by the Filter Control.
The Feature File's filename is shown at the top of the Features View.
The Features View shows \gls{Path} entries or \gls{Histogram} entries,
depending on the type of the Feature File selected in the Reports View.

Use the Features View as follows:
\begin{compactitem}
\item Click on a Feature entry in the Features View
to select the given Feature.
If the selection is a Feature entry, the Feature is Navigated to in the Navigation Area.
If the selection is a Histogram entry,
the set of Features in the Referenced Features View is set
to show all Features matching the selected Histogram entry.
\item Drag on one or more Features to select them.
Once selected, you may copy them using menu command \texttt{Edit | Copy}.
\end{compactitem}

\subsubsection{Referenced Features View}
The Referenced Features View
shows the Feature entries from the Feature File
associated with the Histogram Feature File selected.
The Referenced Features View is only used
when the selected Feature File is a Histogram Feature File.
The referenced Feature File's filename is shown at the top of the Referenced Features View.

All Features associated with a Histogram file are displayed in the Referenced Features View
when a Histogram file is selected in the Reports View.
When a Histogram entry is selected in the Features View,
the Referenced Features View is changed to only show Feature entries
associated with the Histogram entry selected.

Use the Referenced Features View as follows:
\begin{compactitem}
\item Click on a Feature in the Referenced Features View
to \gls{Navigate} to that Feature in the Navigation Area.
\item Drag on one or more referenced Features to select them.
Once selected, you may copy them using menu command \texttt{Edit | Copy}.
\end{compactitem}

\subsection{Navigation Area}
The Navigation Area shows Features that have been Navigated to.

\subsubsection{Navigation Control}
The Navigation Control area shows the Feature that is currently Navigated to,
specifically, the Feature text, Image File, Feature File, 
and Feature Path that are associated with the Feature.
Navigation controls provide Navigation and Bookmarking capabilities.
The Navigation bar shows the currently Navigated Feature
and can be expanded to show the list of Features that have been Navigated to.

Use Navigation controls as follows:
\begin{compactitem}
\item Click on the \gls{NavigationHistory} arrow to open a scrollable list of Features
that have already been Navigated to.
Then click on a Feature in this list to Navigate to that Feature.
\item Click on the Clear button \button{Delete} to clear the currently Navigated Feature
from the Navigation history and to clear the current Navigation.
\item Click on the Bookmark button \button{Bookmark}
to Bookmark the Feature that is currently Navigated to so that it may be exported.
\end{compactitem}

\subsubsection{Highlight Control}
The Highlight Control area manages what content is highlighted in the Feature and Image Views.
The text that is highlighted can be
either the Feature text from the Feature that is currently Navigated to,
or text that you type.

Control Highlighting as follows:
\begin{compactitem}
\item Click on the \gls{HighlightSource} selector button to control the highlight source.
When the button is pointing up \button{Up},
the Feature text that is currently Navigated to is highlighted,
and text and capitalization case inputs are disabled.
When pointing to the right \button{Forward},
you may input your own text that you would like to be highlighted,
and you can control capitalization case sensitivity.
\item When highlighting your own text,
type text in the text entry field to provide the text that you want highlighted.
\item When highlighting your own text,
you may click the capitalization case button to ignore \button{CaseInsensitive}
or honor \button{CaseSensitive} capitalization case sensitivity.
Note that capitalization case sensitivity is always honored when highlighting Feature text.
\end{compactitem}

\subsubsection{Image View}
The Image View area displays a page of the Image media.
The page of Image media shown is set by Navigating to a Feature.
Once a Feature is Navigated to,
the Image View may be paged forward or backward.

Use the Image View as follows:
\begin{compactitem}
\item Drag on one or more lines in the Image View to select them.
Once selected, you may copy them using menu command \texttt{Edit | Copy}.
\item Click on the \texttt{Text} or \texttt{Hex} buttons to view the Image as text or as binary hexadecimal bytes.
\item Click the Reverse button \button{Back}
to scroll the Image View backward one page.
\item Click the Forward button \button{Forward} to scroll the Image View forward one page.
\item Click the Home button \button{Home}
to scroll back to the Feature that is currently Navigated to.
\end{compactitem}

\section{Opening Reports}
When \bulk runs it creates a new Report directory containing Feature Files, Histogram Files,
and a Report File.
Please refer to Section~\ref{bulk-extractor-output}
for a description of \bulk output and how \bev uses it.

In order for \bev to access Reports, Reports must be opened for browsing.
Opening a Report consists of 1) specifying the path to the Report file that \bulk created
and 2) specifying the path to the Image file that is associated with it.
Open a Report in \bev using menu \texttt{File | Open Report}
which opens a dialog window as shown in Figure~\ref{open-report}.
\begin{figure}
\center
\includegraphics[scale=0.5]{OpenReport}
\caption{Opening a Report consists of
locating the Report file and the Image file that is associated with it.\label{open-report}}
\end{figure}
Note that the Image File may not always be available or may have been moved.
If it is available, select \texttt{Use path from Report}.
If it is moved, enter a custom path.
If it is not available, select \texttt{No Image File},
in which case Image pages will not be shown in the Image View.

\section{Navigating to Features}
Once you have opened one or more Reports, you may Navigate to Features.
Figure~\ref{example-view}
shows an example view with Reports opened, Feature Files loaded, and a Feature navigated to.
Here are some characteristics of this Navigation:
\begin{compactitem}
\item The Report in directory \texttt{block4} has been created using \bulk
and has been opened in \bev.
\item The \texttt{block4} directory outline is expanded,
showing Feature Files that were generated by \bulk.
\item Feature File \texttt{email\_histogram.txt} has been selected in the Reports View
so that its histogram entries show up in the Features View
and the referenced Features in file \texttt{email.txt} show up in the Referenced Features View.
\item Feature \texttt{appro@openssl.org} at offset 11455107 has been selected
so that it is Navigated to in the Navigation area.
\item Because Highlighting is set to highlight Feature text,
the selected Feature is highlighted in the Features Views and in the Image Views.
\item Because no Histogram entry in the Features View has been selected,
the Referenced Features View shows all Features instead of Features associated with a Histogram.
\item Because the Text button is selected in the Navigation area,
the Image page showing in the Image View displays text rather than hexadecimal bytes.
\end{compactitem}

\section{Managing Bookmarks}
Bookmarks provide the ability to generate output for a Case.
Export Bookmarked Features using menu selection \texttt{File | Export Bookmarks}.
Figure~\ref{export-bookmarks}
\begin{figure}
\center
\includegraphics[scale=0.5]{ExportBookmarks}
\caption{An example view of the Export Bookmarks window
showing two Bookmarked Features.\label{export-bookmarks}}
\end{figure}
shows an example Export Bookmarks window with two Bookmarked Features set
and one Bookmarked Feature selected.

The Export Bookmarks window provides the following controls:
\begin{compactitem}
\item The Bookmarks Filename text entry field specifies the filename
to export the Bookmarked Features to.
\item The File Format selector selects the Bookmark output format, and can be Text
for text output, or DFXML for XML output which is compatible with the DFXML format.
\item The \texttt{Clear} button clears the Bookmarked Features list.
\item The \texttt{Delete} button removes a selected Bookmarked Feature from the Bookmarks List.
\item The \texttt{Navigate} button Navigates to a selected Bookmarked Feature in the Bookmarks List.
\item The \texttt{Export} button exports all Bookmarked Features
in the Bookmarks List to the specified Bookmarks file.
\item The \texttt{Close} button closes the Export Bookmarks window.
\end{compactitem}
The Text Bookmarks file generated from the two Bookmarks in this example is shown
in Figure~\ref{text-bookmarks}.
\begin{figure}
\small
\begin{verbatim}

mtdblock4.img, 11455107, appro@openssl.org
11452416  .a..z..7...Y.<...'..5.a.....<.GzY...?s.Uy....7.s...S[.._.o=...Dx
11452480  ....>.h.,4$8_@..r....%...I<(A...q..9..........Vda..{p.2.t\lHBW..
11452544  R.j.06.8.@......|.9../..4.CD....T{.2..#=.L..B..N...f(.$.v[.Im..%
<lines deleted>
11453440  "t......"....p......",........ ...(..(%..h&.#x..#........p......
11453504  ....#<.... ...!..$)..<&......@...P...`...p.... ...!.. "..0#...J.
11453568  ....AES for ARMv4, CRYPTOGAMS by <appro@openssl.org>...... ... .
11453632  ._-..#......p.......<.M.e_..fo..g.........Q...Q.gq........Q.....
11453696  .}.......p....%.....k...f.+..p........Q...Q.fa........Q......m..
<lines deleted>
11454912  ..*......=....*...*......0......f...f...g........0....>.....L...
11454976  P.M...].T........]...0...@..eQ..fa..gq........1......_..........
11455040  ../..y.Z...n......b.SHA1 block transform for ARMv4, CRYPTOGAMS b
11455104  y <appro@openssl.org>............/.B.D7q........[.V9...Y..?..^..
11455168  .....[....1$.}.Ut].r........t....i...G.........$o,.-..tJ...\...v
11455232  RQ>.m.1..'....Y.....G...Qc..g))....'8!...m,M..8STs.e..jv.....,r.
<lines deleted>
11456320  ..... .......@.......@...0....... .......4...8...<...... 0..h...
11456384  .. ... ..0... ).. ... "..0...0...0..d.....+.d.+.......... ......
11456448  .....p.......0....... .......4...8...<......$0..g..... ... ..0..

mtdblock4.img, 54469012-ZIP-407, Htc_Cert@htc.com
    0  0.....*.H..........0......1.0...+......0...*.H..........0...0...
   64  ...........cKA.30...*.H........0..1.0...U....TW1.0...U....Taiwan
  128  1.0...U....Taoyuan City1.0...U....HTC Corporation1.0...U....SSD1
  192  .0...U....HTC Cert1.0...*.H........Htc_Cert@htc.com0...090504193
  256  931Z..360919193931Z0..1.0...U....TW1.0...U....Taiwan1.0...U....T
  320  aoyuan City1.0...U....HTC Corporation1.0...U....SSD1.0...U....HT
  384  C Cert1.0...*.H........Htc_Cert@htc.com0.. 0...*.H.............0
  448  ..........Kj.#U..1.jB|.....w.p...V^..S...[B........D.U.cg..Q....
  512  c.U.......}.@S..8.5J}.<(.U.ax,.....h`........;h.>u.;......s.<q.o
<lines deleted>
 1600  ....lY\..v7.y..d...5..Y....`...M>/J..gL.d.M(8.._.7L).1....7./,w6
 1664  ]...L...=W..a........U0.z...X.........
\end{verbatim}
\normalsize
\caption{An excerpt of the Text output generated from the two Bookmarks
in the example Bookmarks List.\label{text-bookmarks}}
\end{figure}
Note that there is less output from the second Bookmark
because the second Feature is from a path using zip compression
and contains fewer bytes than an Image page can display.

\section{Managing Cases}
A Case consists of the information related to an investigation.
For \bev, this information consists of the set of Reports that are currently opened
and the set of Bookmarks that are currently marked.

Cases can be opened and saved as Case files.
Case files simply contain Reports and Bookmarked Features in XML format
that is compatible with DFXML.
Case files are useful because they wrap up the state of an investigation
so that it can be moved to another system or shared with another investigator.
Save a Case to a file as shown in Figure~\ref{save-case}
using menu command \texttt{File | Save Case}.
\begin{figure}
\center
\includegraphics[scale=0.5]{SaveCase}
\caption{A Case, specifically the currently active set of Reports and Bookmarks,
is saved to a Case file.\label{save-case}}
\end{figure}
Open a Case from a file as shown in Figure~\ref{open-case}
using menu command \texttt{File | Open Case}.
\begin{figure}
\center
\includegraphics[scale=0.5]{OpenCase}
\caption{A Case is opened from a saved Case file.\label{open-case}}
\end{figure}
Transfer Case files between systems just as you transfer any other file.
Use option \texttt{Keep Existing Case Open} to open a Case
without removing data from the currently open environment.
This option augments existing settings with Case settings
rather than replacing them.

\section{Examples}
\subsection{Obtaining a Report}
The Bulk Extractor Viewer views Features
from Feature Files in Report directories generated by \bulk.
Create new Reports using \bulk
or copy existing Reports generated by \bulk into your file system.

\subsubsection{Creating a new Report}
Create a new Report by running \bulk from within \bev:
\begin{compactenum}
\item Click Menu item \texttt{File | Run bulk\_extractor}.
\item Enter the Image file for \bulk to scan.
Enter the new output directory path for the Report that \bulk will generate.
Set other \bulk parameters, as desired \cite{garfinkel:bulk-extractor}.
\item Once parameters are set, click the \texttt{Start bulk\_extractor} button.
\item It typically takes multiple hours to read Image data from a disk drive.
Monitor the status of the scan in the \bulk Scan window.
This window will indicate when the scan is done and the Report is complete.
\end{compactenum}
Please see Section~\ref{running-bulk-extractor} for more information on running \bulk.

\subsubsection{Copying in an existing Report}
Copy (or mount) the Report directory into your file system.
If the Image media is available,
copy (or mount) the Image into your file system too.
If the Image file will be at a different path from where it was created,
you will need to select the alternate Image path when opening the Report.
If the Image file is not available,
you will need to select \texttt{No Image File} when opening the Report.

For convenience, an existing demo Report and Image file are available
online at \url{http://domex.nps.edu/deep/Bulk\_Extractor.html}.
The Report directory and its files
are at \url{http://domex.nps.edu/deep/begui/nps-2010-emails\_e01\_dir.zip}
and the Image file
is at \url{http://digitalcorpora.org/corp/drives/nps/nps-2010-emails/nps-2010-emails.E01}.
You will need to unzip the Report directory into a new directory before using it.

\subsection{Opening a Report}
When \bulk has completed its scan and generated a Report,
use \bev to open the Report and view its Features:

\begin{compactenum}
\item Click on Menu item \texttt{File | Open Report} to open the \texttt{Open Report} window.
\item Select the desired Report file (report.xml) using the file chooser.
Keep the Image File selector at \texttt{Use path from Report}
unless the Image file has moved to a different path or is not available.
\item Click \texttt{OK}.
The Report shows up in the Reports area and is available for browsing.
\end{compactenum}

\subsection{Navigating to Features}
Features may be navigated to by clicking on Feature entries in a Report,
clicking on a Feature in the Navigation history,
or clicking on a Bookmarked Feature.
\subsubsection{Navigating to a Feature in a Report}
\begin{compactenum}
\item Open a Report if one is not open already.
\item Click on a Feature File in the Reports View
to view its Features in the Features View.
\item Click on a Feature line to Navigate to that Feature.
If the Feature File has Feature entries (the file is not a Histogram file),
Feature entries will be in the Features View.
If the Feature File has Histogram entries (the file is a Histogram file),
Feature entries will be in the Referenced Features View.
\end{compactenum}

\subsubsection{Navigating to a Feature in the Navigation History}
Click on a Feature in the scrollable Feature history list in the Navigation Control area
to navigate to it.

\subsubsection{Navigating to a Bookmarked Feature}
Bookmarked features may also be Navigated to:
\begin{compactenum}
\item Open the Export Bookmarks Window using menu command \texttt{File | Export Bookmarks}.
\item Select a Bookmarked Feature in the Bookmarks List.
\item Click on the \texttt{Navigate} button to navigate to the Bookmarked Feature
selected in the Bookmarks List.
\end{compactenum}

\subsection{Preparing a Forensics Report}
Prepare a Forensics report by Bookmarking specific Features
and then exporting the Bookmarks to a file:
\begin{compactenum}
\item For each Feature to be in the Report:
\begin{compactenum}
\item Navigate to that Feature.
\item Bookmark it by clicking the Bookmark button \button{Bookmark}
while it is Navigated to.
\end{compactenum}
\item Open the Export Bookmarks Window using menu command \texttt{File | Export Bookmarks}.
\item Review the Bookmarks List in the Export Bookmarks Window:
\begin{compactenum}
\item Click Bookmark entries in the Bookmarks List to select them.
\item Click the \texttt{Navigate} button to review Bookmarked Features
and verify that they are desired for export.
\item Click the \texttt{Delete} button to delete unwanted Bookmarked Feature entries.
\end{compactenum}
\item Enter the filename for the new text file that will contain the exported Bookmarks.
\item Choose the Text file format to select \texttt{Text} output rather than \texttt{DFXML} output.
\item Click the \texttt{Export} button.
A progress indicator will appear if the Export process takes time.
It will go away when the export process is complete.
\item Print the newly created Bookmarks file.
\end{compactenum}

\subsection{Managing Cases}
We define a Case as the Reports and Bookmarked Features that get set
while performing an examination.
\bev allows us to save and open Case information
so that we can reload them later
or transfer them to another system.
An example workflow for managing Cases follows:
\begin{compactenum}
\item Open one or more Reports related to a Case.
\item Navigate to and Bookmark Features relevant to the Case.
\item Save the Case using menu command \texttt{File | Save Case}.
\item Open the Case at a later time or on another system
using menu command \texttt{File | Open Case}
and specifying the saved Case file.
When opening a Case,
you may augment existing Case settings by selecting option \texttt{Keep Existing Case Open}
or replace existing Case settings by not selecting option \texttt{Keep Existing Case Open}.
\end{compactenum}
Case files are useful for opening and saving Cases and do not contain actual Image data.
Case files are created in an XML format compatible with DFXML.

\section{Running \bulk\label{running-bulk-extractor}}
\bev provides User Interfaces for generating Reports using \bulk.
\subsection{Setting \bulk Parameters}
Set parameters for running \bulk
as shown in the window in Figure~\ref{run-bulk-scan-options}.
\begin{figure}
\center
\includegraphics[scale=0.5]{RunBulkScanOptions}
\caption{The \bev provides a User Interface for starting a \bulk scan.\label{run-bulk-scan-options}}
\end{figure}
Open this window using menu selection \texttt{Tools | Run bulk\_extractor}.
This window is split into two tabs: a Scan Options tab and a Processing Options tab.
The Scan Options in the first tab are split into several categories:
Required parameters, General Options, Tuning Parameters, Scanners, and Plugins.

\emph{Required parameters} define the input Image file and output directory that \bulk will use.
These are the only parameters that are required by \bulk.
When starting a new scan, the output directory must not exist yet.
If restarting an interrupted scan,
the output directory may exist, but it must be valid for continuing the interrupted scan.

\emph{General options} are used for defining parameters that control the output.

\emph{Tuning parameters} provide miscellaneous control.

\emph{Scanners} controls which scanners will be run.

\emph{Plugins} provide a means to run custom plugins.

The Processing Options in the second tab provide a means for customizing debugging.
Processing Options are shown in Figure~\ref{run-bulk-processing-options}.
\begin{figure}
\center
\includegraphics[scale=0.5]{RunBulkProcessingOptions}
\caption{The second tab in the User Interface for starting a \bulk scan
provides controls for debugging.\label{run-bulk-processing-options}}
\end{figure}

Please see \bulk documentation \cite{garfinkel:bulk-extractor}
for complete information on using \bulk.

\subsection{Starting \bulk}
Once all the inputs are set, start the scan process
by clicking the \texttt{Start bulk\_extractor} button.
The scan process typically takes hours to complete for large Images.

\subsection{Monitoring the Progress of the \bulk Scan}
Once started, a progress window opens as shown in Figure~\ref{bulk-extractor-running}.
\begin{figure}
\center
\includegraphics[scale=0.5]{BulkExtractorRunning}
\caption{An example \bulk scan window
showing \bulk at four percent complete.\label{bulk-extractor-running}}
\end{figure}

The progress window includes the following:
\begin{compactitem}
\item The name of the Image file that \bulk will scan
and the name of the Feature directory that \bulk will write to.
\item An animated progress indicator showing percent complete.
\item The input parameters provided when starting \bulk.
\item Any status information reported by \bulk.
\item A \texttt{Cancel} button for aborting the scan.
\end{compactitem}

\subsection{\bulk Output\label{bulk-extractor-output}}
\bulk creates a Report directory containing Feature Files, Histogram Files, and a Report File.
\subsubsection{Feature Files}
\bulk creates a number of \glspl{FeatureFile}, one for each Feature type scanned,
containing Features identified in the Image.
Some example Feature File filenames are
\texttt{email.txt}, \texttt{telephone.txt}, and \texttt{ip.txt}.

Feature entries in a Feature File contain the Feature text and the \gls{Path} in the Media
where the Feature resides.
Two types of Paths are supported:
\begin{compactitem}
\item Direct Paths indicate the location of the Feature
with respect to the start of the Image.
\item Compressed Paths define the location of Features
that are embedded within compressed regions such as within a \texttt{.zip} region.
Compressed paths can be recursive.
\end{compactitem}

\bev displays Feature entries in the Features Area of its main window.

\subsubsection{Histogram Files}
\bulk creates \glspl{HistogramFile} for providing \gls{Histogram} information
relating to Features.
An example Histogram filename is \texttt{email\_histogram.txt}.

Histogram entries in a Histogram File contain Feature text and the number of times
the Feature was observed in the Media.
Histogram entries do not contain Path information to where Features reside.

When a Histogram File is selected in the Reports View,
its Histogram entries are displayed in the Feature View
and the \gls{ReferencedFeatureFile} associated with the Histogram File
is displayed in the Referenced Feature View.

\subsubsection{Report File}
\bulk creates \gls{Report} file \texttt{report.xml}
to record facts about the \bulk run that generated the Report.
Some of the facts recorded include:
\begin{compactitem}
\item The path to the Image file that was used.
\item Input parameters provided to the \bulk scan.
\item A summary of the number of Features found.
\item The version of \bulk used to generate the Report.
\end{compactitem}

\section{Managing \bev Views}
Views are managed using commands in the \texttt{View} menu:
\begin{compactitem}
\item \texttt{Highlight Source} selects what text is highlighted in the Feature and Image views,
either Feature text or typed text.  See \emph{Highlight Control} in Section~\ref{main-window-parts}.
\item \texttt{Image Format} selects how Image media is displayed,
specifically as lines of text or as a grid of hexadecimal bytes.
See \emph{Image View} in Section~\ref{main-window-parts}.
\item \texttt{Address Base} selects the address base of the offset values
displayed in the Features View, Referenced Features View, and Image View.
Selectable address base formats are \texttt{Decimal} and \texttt{Hexadecimal}.
\item \texttt{Feature File Font}
sets the font size used in the Features View and Referenced Features View.
\item \texttt{Image File Font}
sets the font size used in the Image View.
\item The \texttt{Referenced Features}
view controls visibility of the Referenced Features View area.
The Referenced Features View is not used unless the Features View is displaying a Histogram file.
It can be set to \texttt{Always Visible}, which is the default,
or \texttt{Collapsed when not Referenced} to hide it when it is not used.
\item The \texttt{Properties} views open windows for showing miscellaneous information:
\begin{compactitem}
\item \texttt{Image Metadata} displays information about the Image
associated with the currently selected Feature.
Image Metadata is not returned unless a Feature is Navigated to.
\item \texttt{Report Properties} displays properties about the Report
associated with the currently selected Feature.
Specifically, file \texttt{report.xml} is displayed.
Report Properties information is not returned unless a Feature is Navigated to.
\end{compactitem}
\end{compactitem}

\section{Using the \texttt{Help} Menu}
The \texttt{Help} Menu provides the following resources:
\begin{compactitem}
\item Version information, useful for tracking version changes.
\item Log information, useful for identifying the cause of a failure.
\item Diagnostics controls,
which are used to facilitate testing and are not intended for general use.
\end{compactitem}

\section{Reporting Errors}
If anything fails, please email a report of the failure
to the Developer, Bruce Allen, at \href{mailto:bdallen@nps.edu}{bdallen@nps.edu}.

Please provide the following information:
\begin{compactitem}
\item Version information:
\begin{compactitem}
\item The version number of BEViewer,
obtained from menu command \texttt{Help | About Bulk Extractor Viewer}
and present in the runtime Log.
\item The download source used to install Bulk Extractor.
For Windows, this is typically Windows Installer \texttt{be\_installer.msi}.
If the download source is not \texttt{be\_installer.msi},
please indicate the \bulk version used and how it was installed
or that it is not installed.
\item The Operating System you are using, typically Windows, Linux, or Macintosh.
\end{compactitem}

\item Description:
\begin{compactitem}
\item The observed behavior.
\item The expected behavior.
\item Context such as inputs leading up to the error.
\item If helpful, please also include a screen snapshot
and a description of what should be showing instead.
\end{compactitem}

\item Runtime Log:
\begin{compactitem}
\item If possible, please include a copy of the runtime Log
by copying it
using menu command \texttt{Help | Log Report | Copy to System Clipboard}
and then pasting it into the email.
\end{compactitem}
\end{compactitem}
 
\section{Technical Support}
A Bulk Extractor Users Group is available at \url{bulk\_extractor-users@googlegroups.com}.
This Users Group exists to promote discussions about the use of \bulk and \bev
and to identify future functionality that may benefit Forensics examiners.

For technical support, please email the Bulk Extractor User Group
at \href{mailto:bulk\_extractor-users@googlegroups.com}{bulk\_extractor-users@googlegroups.com}
or contact the Developer, Bruce Allen, at \href{mailto:bdallen@nps.edu}{bdallen@nps.edu}.

Please post new Feature recommendations, change requests,
or usability comments to the Bulk Extractor User Group or to the Developer.
Please also browse the list of proposed Future Work
posted under \url{http://domex.nps.edu/deep/Bulk\_Extractor.html}.


\cleardoublepage
\bibliography{beviewer-um}
\bibliographystyle{plain}

\cleardoublepage
\printglossaries

\begin{comment}
\section{Vocabulary}
\bev
\bulk
Feature
Feature File
Histogram File
Histogram
Referenced Feature File
Referenced Features
Bookmarked Features
Report
Case
Navigate
Navigation History
Path
Offset
Image File
Image
Address Base
Highlight
Highlight Source
Filter
Log
System Clipboard
Image Reader
\end{comment}

\end{document}

